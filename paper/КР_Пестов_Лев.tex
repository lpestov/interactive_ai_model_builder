\documentclass[14pt]{extarticle}
\usepackage[T2A]{fontenc}
\usepackage[utf8]{inputenc}
\usepackage[left=2cm,right=2cm, top=2cm,bottom=2cm]{geometry}
\usepackage[english,russian]{babel}
\usepackage[pdftex,unicode=true,colorlinks,filecolor=black,citecolor=black,linkcolor=black]{hyperref}
\setcounter{secnumdepth}{5}

\begin{document}
\begin{titlepage}
\begin{center}
\normalsize
\normalsize{ФЕДЕРАЛЬНОЕ ГОСУДАРСТВЕННОЕ АВТОНОМНОЕ\\ОБРАЗОВАТЕЛЬНОЕ УЧРЕЖДЕНИЕ\\ВЫСШЕГО ОБРАЗОВАНИЯ\\«НАЦИОНАЛЬНЫЙ ИССЛЕДОВАТЕЛЬСКИЙ УНИВЕРСИТЕТ\\

ВЫСШАЯ ШКОЛА ЭКОНОМИКИ»}

\vfill

\textbf{Факультет информатики, математики и компьютерных наук}\\[3mm]

\textbf{Программа подготовки бакалавров по направлению\\Компьютерные науки и технологии}

\vfill

\textit{Пестов Лев Евгеньевич}\\[3mm]

\textbf{КУРСОВАЯ РАБОТА}\\[10mm]

\normalsize{ 
 	
Интерактивный конструктор моделей искусственного интеллекта с поддержкой мультимодальных задач. \\Реализация пайплайна обучения глубоких нейросетей}

\end{center}

\vfill
\newlength{\ML}
\settowidth{\ML}{«\underline{\hspace{0.7cm}}»
\underline{\hspace{2cm}}}
\hfill
\begin{minipage}{0.4\textwidth}
\raggedleft{Научный руководитель\\
старший преподаватель НИУ ВШЭ - НН\\[2mm]
Саратовцев Артём Романович}
\end{minipage}%
\vfill
\begin{center}

Нижний Новгород, 2025г.

\end{center}
\end{titlepage}
\end {document}
